% Options for packages loaded elsewhere
\PassOptionsToPackage{unicode}{hyperref}
\PassOptionsToPackage{hyphens}{url}
%
\documentclass[
]{article}
\usepackage{lmodern}
\usepackage{amssymb,amsmath}
\usepackage{ifxetex,ifluatex}
\ifnum 0\ifxetex 1\fi\ifluatex 1\fi=0 % if pdftex
  \usepackage[T1]{fontenc}
  \usepackage[utf8]{inputenc}
  \usepackage{textcomp} % provide euro and other symbols
\else % if luatex or xetex
  \usepackage{unicode-math}
  \defaultfontfeatures{Scale=MatchLowercase}
  \defaultfontfeatures[\rmfamily]{Ligatures=TeX,Scale=1}
\fi
% Use upquote if available, for straight quotes in verbatim environments
\IfFileExists{upquote.sty}{\usepackage{upquote}}{}
\IfFileExists{microtype.sty}{% use microtype if available
  \usepackage[]{microtype}
  \UseMicrotypeSet[protrusion]{basicmath} % disable protrusion for tt fonts
}{}
\makeatletter
\@ifundefined{KOMAClassName}{% if non-KOMA class
  \IfFileExists{parskip.sty}{%
    \usepackage{parskip}
  }{% else
    \setlength{\parindent}{0pt}
    \setlength{\parskip}{6pt plus 2pt minus 1pt}}
}{% if KOMA class
  \KOMAoptions{parskip=half}}
\makeatother
\usepackage{xcolor}
\IfFileExists{xurl.sty}{\usepackage{xurl}}{} % add URL line breaks if available
\IfFileExists{bookmark.sty}{\usepackage{bookmark}}{\usepackage{hyperref}}
\hypersetup{
  pdftitle={ShOpt.jl \textbar{} A Julia Library for Empirical Point Spread Function Characterization of JWST NIRCam Images},
  hidelinks,
  pdfcreator={LaTeX via pandoc}}
\urlstyle{same} % disable monospaced font for URLs
\usepackage{color}
\usepackage{fancyvrb}
\newcommand{\VerbBar}{|}
\newcommand{\VERB}{\Verb[commandchars=\\\{\}]}
\DefineVerbatimEnvironment{Highlighting}{Verbatim}{commandchars=\\\{\}}
% Add ',fontsize=\small' for more characters per line
\newenvironment{Shaded}{}{}
\newcommand{\AlertTok}[1]{\textcolor[rgb]{1.00,0.00,0.00}{\textbf{#1}}}
\newcommand{\AnnotationTok}[1]{\textcolor[rgb]{0.38,0.63,0.69}{\textbf{\textit{#1}}}}
\newcommand{\AttributeTok}[1]{\textcolor[rgb]{0.49,0.56,0.16}{#1}}
\newcommand{\BaseNTok}[1]{\textcolor[rgb]{0.25,0.63,0.44}{#1}}
\newcommand{\BuiltInTok}[1]{#1}
\newcommand{\CharTok}[1]{\textcolor[rgb]{0.25,0.44,0.63}{#1}}
\newcommand{\CommentTok}[1]{\textcolor[rgb]{0.38,0.63,0.69}{\textit{#1}}}
\newcommand{\CommentVarTok}[1]{\textcolor[rgb]{0.38,0.63,0.69}{\textbf{\textit{#1}}}}
\newcommand{\ConstantTok}[1]{\textcolor[rgb]{0.53,0.00,0.00}{#1}}
\newcommand{\ControlFlowTok}[1]{\textcolor[rgb]{0.00,0.44,0.13}{\textbf{#1}}}
\newcommand{\DataTypeTok}[1]{\textcolor[rgb]{0.56,0.13,0.00}{#1}}
\newcommand{\DecValTok}[1]{\textcolor[rgb]{0.25,0.63,0.44}{#1}}
\newcommand{\DocumentationTok}[1]{\textcolor[rgb]{0.73,0.13,0.13}{\textit{#1}}}
\newcommand{\ErrorTok}[1]{\textcolor[rgb]{1.00,0.00,0.00}{\textbf{#1}}}
\newcommand{\ExtensionTok}[1]{#1}
\newcommand{\FloatTok}[1]{\textcolor[rgb]{0.25,0.63,0.44}{#1}}
\newcommand{\FunctionTok}[1]{\textcolor[rgb]{0.02,0.16,0.49}{#1}}
\newcommand{\ImportTok}[1]{#1}
\newcommand{\InformationTok}[1]{\textcolor[rgb]{0.38,0.63,0.69}{\textbf{\textit{#1}}}}
\newcommand{\KeywordTok}[1]{\textcolor[rgb]{0.00,0.44,0.13}{\textbf{#1}}}
\newcommand{\NormalTok}[1]{#1}
\newcommand{\OperatorTok}[1]{\textcolor[rgb]{0.40,0.40,0.40}{#1}}
\newcommand{\OtherTok}[1]{\textcolor[rgb]{0.00,0.44,0.13}{#1}}
\newcommand{\PreprocessorTok}[1]{\textcolor[rgb]{0.74,0.48,0.00}{#1}}
\newcommand{\RegionMarkerTok}[1]{#1}
\newcommand{\SpecialCharTok}[1]{\textcolor[rgb]{0.25,0.44,0.63}{#1}}
\newcommand{\SpecialStringTok}[1]{\textcolor[rgb]{0.73,0.40,0.53}{#1}}
\newcommand{\StringTok}[1]{\textcolor[rgb]{0.25,0.44,0.63}{#1}}
\newcommand{\VariableTok}[1]{\textcolor[rgb]{0.10,0.09,0.49}{#1}}
\newcommand{\VerbatimStringTok}[1]{\textcolor[rgb]{0.25,0.44,0.63}{#1}}
\newcommand{\WarningTok}[1]{\textcolor[rgb]{0.38,0.63,0.69}{\textbf{\textit{#1}}}}
\setlength{\emergencystretch}{3em} % prevent overfull lines
\providecommand{\tightlist}{%
  \setlength{\itemsep}{0pt}\setlength{\parskip}{0pt}}
\setcounter{secnumdepth}{-\maxdimen} % remove section numbering
\newlength{\cslhangindent}
\setlength{\cslhangindent}{1.5em}
\newenvironment{cslreferences}%
  {\setlength{\parindent}{0pt}%
  \everypar{\setlength{\hangindent}{\cslhangindent}}\ignorespaces}%
  {\par}

\title{ShOpt.jl \textbar{} A Julia Library for Empirical Point Spread
Function Characterization of JWST NIRCam Images}
\author{}
\date{23 August 2023}

\begin{document}
\maketitle

\hypertarget{summary}{%
\section{Summary}\label{summary}}

When cosmologists try to take pictures of space, a combination of the
photometry of the camera and atmospheric affects distort the light that
comes from stars. Stars are examples of what Astronomers call point
sources, and so the aptly named Point Spread Function (PSF) is a
mathematical model that quantifies exactly how the light is being
distorted. The point spread function takes as input a delta function and
a position and outputs a lensed image. The goal of empirical point
spread function characterization is to be able to point to any position
on your camera and predict what the lensed star looks like. Once we have
a model that can do this well, we can deconvolve our images with the
point spread function to obtain what the image would look like in the
absense of lensing. The empirical way to do this is to take our images
of lensed stars and seperate them into training and validation set. Our
point spread function will be found by interpolating the training stars
across the field of view of the camera and validated by comparing the
reserved stars to the point spread function's prediction.

Shear Optimization with \texttt{ShOpt.jl} introduces modern techniques
for empirical Point Spread Function characterization across the Full
Field of View tailored to the data captured by the James Webb Space
Telescope. To first order, we can approximate our images with analytic
profiles. We adopt a multivariate gaussian because it is cheap to fit to
an image. This function is parameterized by three variables,
\([s, g_1, g_2]\), where \(s\) corresponds to size and \(g_1 , g_2\)
correspond to shear. After we fit this function to our stars with
\texttt{Optim.jl} and \texttt{ForwardDiff.jl} (Mogensen and Riseth 2018;
Revels, Lubin, and Papamarkou 2016), we interpolate the parameters
across the field of view according to position. Essentially, each star
is a datapoint, and the three variables are given polynomials in focal
plane coordinates of degree \(n\), where \(n\) is supplied by the user.
For a more precise model, we also give each pixel in our images a
polynomial and interpolate it across the field of view. This is referred
to in the literature as a pixel grid fit (Jarvis et al. 2020).

\texttt{ShOpt.jl} takes inspiration from a number of algorithms outside
of astronomy. Mainly, SE-Sync (Rosen et al. 2019), an algorithm that
provides a certifiably correct solution to a robotting mapping problem
by considering the manifold properites of the data. We borrow this idea
to put a constraint on the solutions we obtain to \([s, g_1, g_2]\).
(Bernstein and Jarvis 2002) outlined the manifold properties of shears
for us, so we knew from the get go that our solution was constrained to
the manifold \(B_2(r) \times \mathbb{R}_{+}\). While it was known that
this constrain existed in the literature, the parameter estimation
tasked had been framed as an unconstrained problem prior to our work
(Jarvis et al. 2020). For a more rigorous treatment of optimization on
manifolds see (Absil, Mahony, and Sepulchre 2008) and (Boumal 2023).
\texttt{Julia} has lots of support for working with manifolds with
\texttt{Manopt}, which we may leverage in future releases (Bergmann
2022).

\texttt{ShOpt.jl} provides two modes for pixel grid fits,
\texttt{PCA\ mode} and \texttt{Autoencoder\ mode}. Each mode provides
the end user with tunable parameters that allow for both perfect
reconstruction of the model vignets and low dimensional representations.
The advantage of these modes is that they provide good reconstructions
of the lensed images while fixating on the actual star and not the
background noise. In this way it generates a datapoint for our empirical
point spread function to learn and denoises the image in one step.

\texttt{PCA\ mode}, outlined here, reconstructs it's images using the
first n principal components.

\begin{Shaded}
\begin{Highlighting}[]
\KeywordTok{function}\NormalTok{ pca\_image(image}\OperatorTok{,}\NormalTok{ ncomponents)    }
  \CommentTok{\#Load img Matrix}
\NormalTok{  img\_matrix }\OperatorTok{=}\NormalTok{ image}
    
  \CommentTok{\# Perform PCA    }
\NormalTok{  M }\OperatorTok{=}\NormalTok{ fit(PCA}\OperatorTok{,}\NormalTok{ img\_matrix}\OperatorTok{;}\NormalTok{ maxoutdim}\OperatorTok{=}\NormalTok{ncomponents)    }
    
  \CommentTok{\# Transform the image into the PCA space    }
\NormalTok{  transformed }\OperatorTok{=}\NormalTok{ MultivariateStats.transform(M}\OperatorTok{,}\NormalTok{ img\_matrix)    }
    
  \CommentTok{\# Reconstruct the image    }
\NormalTok{  reconstructed }\OperatorTok{=}\NormalTok{ reconstruct(M}\OperatorTok{,}\NormalTok{ transformed)    }
    
  \CommentTok{\# Reshape the image back to its original shape    }
\NormalTok{  reconstructed\_image }\OperatorTok{=}\NormalTok{ reshape(reconstructed}\OperatorTok{,}\NormalTok{ size(img\_matrix)}\OperatorTok{...}\NormalTok{)    }
\KeywordTok{end}    
\end{Highlighting}
\end{Shaded}

\texttt{Autoencoder\ mode} uses a neural network to reconstruct the
image from a lower dimensional latent space. The network code written
with \texttt{Flux.jl} is below (Innes 2018)

\begin{Shaded}
\begin{Highlighting}[]
\CommentTok{\# Encoder    }
\NormalTok{encoder }\OperatorTok{=}\NormalTok{ Chain(    }
\NormalTok{                Dense(r}\OperatorTok{*}\NormalTok{c}\OperatorTok{,} \FloatTok{128}\OperatorTok{,}\NormalTok{ leakyrelu)}\OperatorTok{,}    
\NormalTok{                Dense(}\FloatTok{128}\OperatorTok{,} \FloatTok{64}\OperatorTok{,}\NormalTok{ leakyrelu)}\OperatorTok{,}    
\NormalTok{                Dense(}\FloatTok{64}\OperatorTok{,} \FloatTok{32}\OperatorTok{,}\NormalTok{ leakyrelu)}\OperatorTok{,}    
\NormalTok{               )    }
\CommentTok{\#Decoder}
\NormalTok{decoder }\OperatorTok{=}\NormalTok{ Chain(    }
\NormalTok{                Dense(}\FloatTok{32}\OperatorTok{,} \FloatTok{64}\OperatorTok{,}\NormalTok{ leakyrelu)}\OperatorTok{,}    
\NormalTok{                Dense(}\FloatTok{64}\OperatorTok{,} \FloatTok{128}\OperatorTok{,}\NormalTok{ leakyrelu)}\OperatorTok{,}    
\NormalTok{                Dense(}\FloatTok{128}\OperatorTok{,}\NormalTok{ r}\OperatorTok{*}\NormalTok{c}\OperatorTok{,}\NormalTok{ tanh)}\OperatorTok{,}    
\NormalTok{               )    }
\CommentTok{\#Full autoencoder}
\NormalTok{autoencoder }\OperatorTok{=}\NormalTok{ Chain(encoder}\OperatorTok{,}\NormalTok{ decoder)    }
                      
\CommentTok{\#x\_hat = autoencoder(x)    }
\NormalTok{loss(x) }\OperatorTok{=}\NormalTok{ mse(autoencoder(x)}\OperatorTok{,}\NormalTok{ x)    }
    
\CommentTok{\# Define the optimizer    }
\NormalTok{optimizer }\OperatorTok{=}\NormalTok{ ADAM()    }
\end{Highlighting}
\end{Shaded}

\hypertarget{statement-of-need}{%
\section{Statement of need}\label{statement-of-need}}

While there are many existing empirical PSF fitters, they were created
as apart of the efforts of other collaborations with their own cameras
and science goals. Mainly, The Dark Energy Survey and DESCam (Jarvis et
al. 2020; Flaugher et al. 2015). The recent data from the James Webb
Space Telescope poses new challenges.

\begin{enumerate}
\def\labelenumi{(\arabic{enumi})}
\item
  The James Webb PSFs are not well approximated by analytic profiles.
  This calls for well thought out parametric free models that can
  capture the full dynamic range of the Point Spread Function without
  fixating on the noise in the background.
\item
  The NIRCam detectors measure 0.03"/pix (Gardner et al. 2006). To
  capture an accurate description of the Point Spread Function at this
  scale we need images that are \(131\) by \(131\) to \(261\) by \(261\)
  pixels across. These vignet sizes are much larger in comparison to the
  sizes needed for previous large scale surveys such as DES (Jarvis et
  al. 2020) and SuperBIT (McCleary et al. 2023) and forces us to
  evaluate how well existing PSF fitters scale to this size.
\end{enumerate}

\hypertarget{state-of-the-field}{%
\section{State of the Field}\label{state-of-the-field}}

There are several existing empirical PSF fitters in addition to a
theoretical prediction of the James Webb PSFs developed by STScI (Jarvis
et al. 2020 ; Bertin 2011; Perrin et al. 2014 , 2012). We describe them
here and draw attention to their strenghts and weaknesses to motivate
the development of \texttt{ShOpt.jl}. The first empirical PSF fitter
developed was \texttt{PSFex}. It used statistical methods that were
natural starting points for the problem at hand and prove to be
sufficient in many cases to this day. However, as Mike Jarvis and his
collaborators with DES noticed, \texttt{PSFex} produced a systematic
size bias of the Point Spread Function with how it calculated spatial
variation across the field of view (Jarvis et al. 2020)

\texttt{PIFF} (Point Spread Functions in the Full Field of View)
followed \texttt{PSFex} in the effort to correct this issue. The DES
camera was \(2.2\) degrees across, which was large enough for the size
bias to become noticable for their efforts. \texttt{PIFF} works in focal
plane coordinates as opposed to sky coordinates which fixes the
systematic size bias. Jarvis and DES also used the \texttt{Python}
libraries of astropy (Astropy Collaboration et al. 2022) and Galsim
(Rowe et al. 2015) to make the software more accessible than PSFex.
PSFex was written in \texttt{C} and had been active for more than 20
years before the systematic size bias was discovered. Due to being so
old and written in a low level language it is much less approachable.
One of the motivations of \texttt{ShOpt} was to write astrophysics
specific software in \texttt{Julia}, because \texttt{Julia} provides a
nice balance of readability and speed with it's high level functional
paradigm and just in time compiler.

While we do have theoretical models of the James Webb PSF, there is yet
to be any validation of these models on real data in the literature.
Additionally, these models are for single exposure images. The James
Webb images have both single exposure and mosaiced images (Perrin et al.
2014, 2012). Mosaiced images are essentially single exposure detector
images concatenated together side by side. The PSF models for single
exposures do not generalize to the mosaics, so empirical models are all
we have for those images.

The COMOS-Web survey is the largest extragalactic survey according to
area and prime time allocation (Casey et al. 2023), and takes up
\(0.54 deg^2\) (Beichman et al. 2012; Rieke et al. 2023). This is a
large enough portion of the sky that we should prepare to see a lot of
variation across the field of view. This gives \texttt{ShOpt} the
oppurtunity to validate PIFF's correction for handeling PSF variations
and underscore just how impactful (or not impactful) PSFex's size bias
is.

\hypertarget{acknowledgements}{%
\section{Acknowledgements}\label{acknowledgements}}

This project was made possible by The Northeastern Physics Department
and Northeastern Undergraduate Research and Fellowships via the Physics
Research Co-Op Fellowship and PEAK Ascent Award respectively.

\hypertarget{references}{%
\section*{References}\label{references}}
\addcontentsline{toc}{section}{References}

\hypertarget{refs}{}
\begin{cslreferences}
\leavevmode\hypertarget{ref-AbsMahSep2008}{}%
Absil, P.-A., R. Mahony, and R. Sepulchre. 2008. \emph{Optimization
Algorithms on Matrix Manifolds}. Princeton, NJ: Princeton University
Press.

\leavevmode\hypertarget{ref-2022ApJ}{}%
Astropy Collaboration, Adrian M. Price-Whelan, Pey Lian Lim, Nicholas
Earl, Nathaniel Starkman, Larry Bradley, David L. Shupe, et al. 2022.
``The Astropy Project: Sustaining and Growing a Community-oriented
Open-source Project and the Latest Major Release (v5.0) of the Core
Package'' 935 (2): 167. \url{https://doi.org/10.3847/1538-4357/ac7c74}.

\leavevmode\hypertarget{ref-10.1117ux2f12.925447}{}%
Beichman, Charles A., Marcia Rieke, Daniel Eisenstein, Thomas P. Greene,
John Krist, Don McCarthy, Michael Meyer, and John Stansberry. 2012.
``Science opportunities with the near-IR camera (NIRCam) on the James
Webb Space Telescope (JWST).'' In \emph{Space Telescopes and
Instrumentation 2012: Optical, Infrared, and Millimeter Wave}, edited by
Mark C. Clampin, Giovanni G. Fazio, Howard A. MacEwen, and Jacobus M.
Oschmann Jr., 8442:84422N. International Society for Optics; Photonics;
SPIE. \url{https://doi.org/10.1117/12.925447}.

\leavevmode\hypertarget{ref-Bergmann2022}{}%
Bergmann, Ronny. 2022. ``Manopt.jl: Optimization on Manifolds in
Julia.'' \emph{Journal of Open Source Software} 7 (70): 3866.
\url{https://doi.org/10.21105/joss.03866}.

\leavevmode\hypertarget{ref-Bernstein_2002}{}%
Bernstein, G. M., and M. Jarvis. 2002. ``Shapes and Shears, Stars and
Smears: Optimal Measurements for Weak Lensing.'' \emph{The Astronomical
Journal} 123 (2): 583. \url{https://doi.org/10.1086/338085}.

\leavevmode\hypertarget{ref-2011ASPC}{}%
Bertin, E. 2011. ``Automated Morphometry with SExtractor and PSFEx.'' In
\emph{Astronomical Data Analysis Software and Systems Xx}, edited by I.
N. Evans, A. Accomazzi, D. J. Mink, and A. H. Rots, 442:435.
Astronomical Society of the Pacific Conference Series.

\leavevmode\hypertarget{ref-boumal2023intromanifolds}{}%
Boumal, Nicolas. 2023. \emph{An Introduction to Optimization on Smooth
Manifolds}. Cambridge University Press.
\url{https://doi.org/10.1017/9781009166164}.

\leavevmode\hypertarget{ref-casey2023cosmosweb}{}%
Casey, Caitlin M., Jeyhan S. Kartaltepe, Nicole E. Drakos, Maximilien
Franco, Santosh Harish, Louise Paquereau, Olivier Ilbert, et al. 2023.
``COSMOS-Web: An Overview of the Jwst Cosmic Origins Survey.''
\url{http://arxiv.org/abs/2211.07865}.

\leavevmode\hypertarget{ref-2015AJ}{}%
Flaugher, B., H. T. Diehl, K. Honscheid, T. M. C. Abbott, and others.
2015. ``The Dark Energy Camera.'' \emph{AJ} 150: 150.
\url{https://doi.org/10.1088/0004-6256/150/5/150}.

\leavevmode\hypertarget{ref-Gardner_2006}{}%
Gardner, Jonathan P., John C. Mather, Mark Clampin, Rene Doyon, Matthew
A. Greenhouse, Heidi B. Hammel, John B. Hutchings, et al. 2006. ``The
James Webb Space Telescope.'' \emph{Space Science Reviews} 123 (4):
485--606. \url{https://doi.org/10.1007/s11214-006-8315-7}.

\leavevmode\hypertarget{ref-innes:2018}{}%
Innes, Mike. 2018. ``Flux: Elegant Machine Learning with Julia.''
\emph{Journal of Open Source Software}.
\url{https://doi.org/10.21105/joss.00602}.

\leavevmode\hypertarget{ref-Jarvis_2020}{}%
Jarvis, M, G M Bernstein, A Amon, C Davis, P F Lé get, K Bechtol, I
Harrison, et al. 2020. ``Dark Energy Survey Year 3 Results: Point Spread
Function Modelling.'' \emph{Monthly Notices of the Royal Astronomical
Society} 501 (1): 1282--99.
\url{https://doi.org/10.1093/mnras/staa3679}.

\leavevmode\hypertarget{ref-mccleary2023lensing}{}%
McCleary, Jacqueline E, Spencer W Everett, Mohamed M Shaaban, Ajay S
Gill, Georgios N Vassilakis, Eric M Huff, Richard J Massey, et al. 2023.
``Lensing in the Blue Ii: Estimating the Sensitivity of Stratospheric
Balloons to Weak Gravitational Lensing.'' \emph{arXiv Preprint
arXiv:2307.03295}.

\leavevmode\hypertarget{ref-Mogensen2018}{}%
Mogensen, Patrick K., and Asbjørn N. Riseth. 2018. ``Optim: A
Mathematical Optimization Package for Julia.'' \emph{Journal of Open
Source Software} 3 (24): 615. \url{https://doi.org/10.21105/joss.00615}.

\leavevmode\hypertarget{ref-2014SPIE}{}%
Perrin, Marshall D., Anand Sivaramakrishnan, Charles-Philippe Lajoie,
Erin Elliott, Laurent Pueyo, Swara Ravindranath, and Loic. Albert. 2014.
``Updated point spread function simulations for JWST with WebbPSF.'' In
\emph{Space Telescopes and Instrumentation 2014: Optical, Infrared, and
Millimeter Wave}, edited by Jr. Oschmann Jacobus M., Mark Clampin,
Giovanni G. Fazio, and Howard A. MacEwen, 9143:91433X. Society of
Photo-Optical Instrumentation Engineers (Spie) Conference Series.
\url{https://doi.org/10.1117/12.2056689}.

\leavevmode\hypertarget{ref-2012SPIE}{}%
Perrin, Marshall D., Rémi Soummer, Erin M. Elliott, Matthew D. Lallo,
and Anand Sivaramakrishnan. 2012. ``Simulating point spread functions
for the James Webb Space Telescope with WebbPSF.'' In \emph{Space
Telescopes and Instrumentation 2012: Optical, Infrared, and Millimeter
Wave}, edited by Mark C. Clampin, Giovanni G. Fazio, Howard A. MacEwen,
and Jr. Oschmann Jacobus M., 8442:84423D. Society of Photo-Optical
Instrumentation Engineers (Spie) Conference Series.
\url{https://doi.org/10.1117/12.925230}.

\leavevmode\hypertarget{ref-RevelsLubinPapamarkou2016}{}%
Revels, J., M. Lubin, and T. Papamarkou. 2016. ``Forward-Mode Automatic
Differentiation in Julia.'' \emph{arXiv:1607.07892 {[}cs.MS{]}}.
\url{https://arxiv.org/abs/1607.07892}.

\leavevmode\hypertarget{ref-Rieke_2023}{}%
Rieke, Marcia J., Douglas M. Kelly, Karl Misselt, John Stansberry,
Martha Boyer, Thomas Beatty, Eiichi Egami, et al. 2023. ``Performance of
Nircam on Jwst in Flight.'' \emph{Publications of the Astronomical
Society of the Pacific} 135 (1044): 028001.
\url{https://doi.org/10.1088/1538-3873/acac53}.

\leavevmode\hypertarget{ref-doi:10.1177ux2f0278364918784361}{}%
Rosen, David M, Luca Carlone, Afonso S Bandeira, and John J Leonard.
2019. ``SE-Sync: A Certifiably Correct Algorithm for Synchronization
over the Special Euclidean Group.'' \emph{The International Journal of
Robotics Research} 38 (2-3): 95--125.
\url{https://doi.org/10.1177/0278364918784361}.

\leavevmode\hypertarget{ref-rowe2015galsim}{}%
Rowe, Barnaby, Mike Jarvis, Rachel Mandelbaum, Gary M. Bernstein, James
Bosch, Melanie Simet, Joshua E. Meyers, et al. 2015. ``GalSim: The
Modular Galaxy Image Simulation Toolkit.''
\url{http://arxiv.org/abs/1407.7676}.
\end{cslreferences}

\end{document}
